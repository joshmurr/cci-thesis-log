\documentclass[a4paper]{article}
\usepackage[utf8]{inputenc}
\usepackage[
backend=biber,
style=numeric,
bibencoding=ascii]{biblatex}
\addbibresource{bibliography.bib}


\title{ Working Title }
\date{\today}
\author{Josh Murr}

\begin{document}

\pagenumbering{gobble}
\maketitle
% \newpage
% \pagenumbering{arabic}
% \tableofcontents
% \newpage

\begin{abstract}
% 250 Words
  Research in recent years has continually shown new and unexpected applications for Deep Neural Networks which seem to show no limit to their ability. This is in part down to the ubiquity and capability of modern Graphics or Tensor Processing Units which allow for huge amounts of computation at lightning speed. GPT-3 has shown us that simply making a model larger is one possible way to improve results\cite{2005.14165} which could easily set (or continue) the trend that more data and more compute power are the keys to better models. This may in part be true, but this trend widens the gulf of accessibility and also understanding with cutting edge machine learning research. \textit{Learning to See} by Memo Akten et al\cite{2003.00902} provides a glimpse inside a Conditional Generative Adverserial Network (\textit{cGAN}) in an immediate and impactful way showing it's ability and also it's limitations; in todays climate of uncertainty and disbelief this is an important thing. Getting work like this on more modest devices presents the challenge of shrinking such Deep Learning models and improving their efficiency so that they do not require vast compute power to be effective. In this paper I (we?) explore different methods of model compression and efficient architectures and analyse their respective efficiency gains with a view to making such systems more accessible and interactive.
\end{abstract}

\section{Introduction}
% 500 Words

\section{Background Context}
% 500-1000 Words

\section{Method}
% 500-750 words - how you will approach the problem
% (use well understood methods and don’t roll your own)
% mixed methods fine (quantitative - data and qualitative - surveys, viewpoints)

\section{Results}
% 500-750 words

\section{Discussion}
% As long as you can cope with

\section{Conclusion}
% Sum up what you have learned and where we should go next


\medskip
\printbibliography
\end{document}
